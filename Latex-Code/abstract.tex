This thesis aims to disclose necessary features that need to be included in an interface about health literacy that is particularly suitable for children. 
As \textcite{walker2000screen, sharmin2012effect} mention in their papers, technology has gone through great developmental stages and has improved a lot over the last 20 years. As a consequence, technology has become a part of our everyday lives \autocite{gikas2013mobile, gossen2012search}. As \textcite{walker2000screen} mention in their paper, it is not surprising that schools too hope to use technology for educational purposes.  
Thus, in the course of this study, a prototype in the form a quiz was designed. Since the appearance was not the only thing that had to be adapted to the needs and abilities of children, appropriate content had to be chosen too. As a result, the content of the prototype is health literacy with a focus on nutrition. As the study of \textcite{jordan2015gesundheitskompetenz} showed, sometimes even adults are not informed sufficiently. Therefore, it is essential to teach children about health literacy as soon as possible.\\
The developed prototype and the associated testing have revealed that children prefer engaging and rather gamified applications. Furthermore, the majority of the tested children are adequately informed about proper nutrition.