Das Ziel dieser Arbeit ist es herauszufinden, welche Funktionen für ein Interface über Gesundheitskompetenz notwendig sind, um es besonders gut für Kinder zu gestalten.  
Die Technologie hat in den vergangen 20 Jahren erhebliche Veränderungen durchlebt und Fortschritte gemacht wie \textcite{walker2000screen, sharmin2012effect} in ihrem Paper erwähnen. Das hat zur Folge, dass die Technologie zu einem Teil unseres Alltags geworden ist \autocite{gikas2013mobile, gossen2012search}. Daher ist es keine Überraschung, dass Schulen auch verschiedene Technologien in den Unterricht integrieren wollen wie \textcite{walker2000screen} auch in ihrem Paper beschreiben.  
Im Rahmen dieser Studie wurde daher ein Prototyp in Form eines Quiz entwickelt. Da nicht nur das Erscheinungsbild an die Bedürfnisse und Fähigkeiten der Kinder angepasst werden musste, war es notwendig auch einen passenden Inhalt zu finden. Deshalb wurde die Gesundheitskompetenz mit Schwerpunkt auf Ernährung als Inhalt des Quiz gewählt. Wie die Studie von \textcite{jordan2015gesundheitskompetenz} gezeigt hat, sind oft auch Erwachsene nicht ausreichend informiert. Daher ist es wichtig, dass Kinder so schnell wie möglich über Gesundheitskompetenz unterrichtet werden. \\
Der entwickelte Prototyp und das damit verbundene Usertesting haben gezeigt, dass Kinder interaktive und daher auch gamifizierte Anwendungen bevorzugen. Darüber hinaus ist die Mehrheit der getesteten Kinder gut über eine gesunde Ernährung informiert.
